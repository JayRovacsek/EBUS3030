\documentclass{article}
    \usepackage{url}
    \usepackage{cite}    
    \usepackage{xcolor}
    \usepackage{lscape}
    \usepackage{amssymb}
    \usepackage{titling}
    \usepackage{pdfpages}
    \usepackage{enumitem}
    \usepackage{graphicx}
    \usepackage{hyperref}
    \usepackage{listings}
    \usepackage{fancybox}
    \usepackage{enumerate}
    \usepackage{pdflscape}
    \usepackage{afterpage}
    \usepackage{lstautogobble}
    \usepackage[margin=0.8in]{geometry}
    \usepackage[nottoc,notlot,notlof]{tocbibind}
    \renewcommand\maketitlehookd{\vfill\null}
    \renewcommand\maketitlehooka{\null\mbox{}\vfill}

    \newcommand\backgroundimage{
        \put(-5,0){
        \parbox[b][\paperheight]{\paperwidth}{
        \vfill
        \centering
        \includegraphics[height=\paperheight]{Images/background.jpg}
        \vfill
    }}}

    \newcounter{num}

    \graphicspath{ {Images/} }

    \title{EBUS3030 Assignment 1}
    \author{
        Steven Karmaniolos 
        \texttt{c3160280@uon.edu.au}\\
        Jay Rovacsek
        \texttt{c3146220@uon.edu.au}\\
        Jacob Litherland
        \texttt{c3263482@uon.edu.au}\\
        Edward Lonsdale
        \texttt{c3252144@uon.edu.au}
    }
    \date{\today}
    \hypersetup{
    colorlinks=true,
    linkcolor=black,
    filecolor=magenta,      
    urlcolor=blue,
    citecolor=red,
    linktoc=section,
    }
    \pagenumbering{arabic}

    \newlist{legal}{enumerate}{10}
    \setlist[legal]{label*=\arabic*.}

    \begin{document}
    \AddToShipoutPicture{\backgroundimage}

    \begin{titlingpage}
        \maketitle
    \end{titlingpage}

    \includepdf[height=\paperheight,keepaspectratio,pages={1,2}]{Resources/Assignment1Overview.pdf}

    \section{Datamart Business Notes}
    The following business rules were provided to be used in the context of this assignment:
    \begin{itemize}
        \renewcommand\labelitemi{*}
        \item At BIA all customers interacts are in an online environment, 
        there are no orders outside of electronic.
        \item Returning customers can provide POI information via the web
        interface and look up their record and that will flow with the sale.
        \item The sales associate can complete the order form/sale for the
        client.
        \item Each sale will have a reciept number/id.
        \item A reciept can have many line items.
        \item Each line item can only be for a single item, but the customer can
        purchase multiples of the same item.
        \item Where a customer has multiple line items, any sale with more than
        5 row items (containing at least 5 different items) is provided a
        15\% discount.
        \item The system automatically handles the total for the sale by looking
        up the item, then multiplying the costs per item by number
        purchased, and then should store this final field total as a record
        in the system (but should also be able to see clearly sales that
        were provided a discount.
        \item Item prices can change at any point, and the price the customer
        pays is the amount listed for the item on the sale date. We need to
        keep a record of all item prices historically.
        \item Only 1 BIA sales assistant can be attributed to any receipt.
    \end{itemize}

    \newpage
    \section{Data Model}

    \newpage
    \section{Data Load Process (ETL/ELT)}
        Initial import of the data supplied in the xlsx file generated a very basic table
        that allowed us to analyse the data for potential outliers, confirm the business
        requirements of the data and then create tables from which the data model was derived.
        \\
        The Imported table structure was as follows:
        \begin{center}
            \includegraphics{Images/Initial_Import.png}
        \end{center}
        \subsection{Quality Assurance Processes}
        \subsection{Assumptions and Reasoning}
            \subsubsection{Item Table}
                An assumption of the ItemId never needing to be larger than a smallint
                was followed, as a basic query into the maximum range within the test data
                suggested that the maximum Id that currently existed was 30:
                \begin{verbatim}
    -- Some basic queries for us to determine potential outlier data:
    -- What is the max of each column where datatype is int?
    SELECT MAX(Item_ID) AS 'Max Item_ID'
    FROM Assignment1Data;
                \end{verbatim}

                With the results:
                \begin{verbatim}
    Max Item_ID
    30
                \end{verbatim}
                ItemDescription underwent some size optimisation, as the max datalength 
                that currently existed within the supplied data was 52, and we are to assume
                that into the future more items may be added, a value of 255 should allow
                for a varied range of descriptions.
                \\
                SQL queried to determine to above assumption:
                \begin{verbatim}
    -- Determine current max varchar used in Item_Description
    SELECT MAX(DATALENGTH(Item_Description)) 
    FROM Assignment1Data;
                \end{verbatim}
                We do recognise the requirements for optimisation may not require such measures, and 
                acknowledge that a varchar(max)/text datatype would also be reasonable.
                \vspace{5mm}
                \par\noindent
                ItemPrice while imported as float type was considered too precise for the usecase of 
                a monetary value. While MONEY and derivatives exist in the TSQL ecosphere, there are 
                real concerns of accuracy of the datatype\cite{MoneyIssues}, and therefore we decided for 
                a decimal(19,5) typing\cite{Numeric}.
                \vspace{5mm}
                \par\noindent
                The final Item table structure is reflected as:
                \begin{center}
                    \includegraphics{Images/Item_Table.png}
                \end{center}
    \newpage
    \section{Base Analysis}
        \subsection{Raw Results}

    \newpage
    \section{Executive Summary}

    \newpage
    \section{Assumptions}

    \newpage
    \begin{thebibliography}{9}
        \raggedright
        \bibitem{MoneyIssues}
            Reasons against TSQL Money type: Stackoverflow User; \textit{SQLMenace}
            \url{https://stackoverflow.com/questions/582797/should-you-choose-the-money-or-decimalx-y-datatypes-in-sql-server}
        \bibitem{Numeric}
            Microsoft TSQL documentation of Decimal/Numeric types
            \url{https://docs.microsoft.com/en-us/sql/t-sql/data-types/decimal-and-numeric-transact-sql?view=sql-server-2017}
    \end{thebibliography}

    \newpage
    \section{Appendix}
    \label{sec:Appendix}

    \end{document}