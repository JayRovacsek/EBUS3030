\documentclass{article}
    \usepackage{url}
    \usepackage{cite}
    \usepackage{float}   
    \usepackage{xcolor}
    \usepackage{lscape}
    \usepackage{amssymb}
    \usepackage{titling}
    \usepackage{pdfpages}
    \usepackage{enumitem}
    \usepackage{graphicx}
    \usepackage{hyperref}
    \usepackage{fancybox}
    \usepackage{fancyvrb}
    \usepackage{enumerate}
    \usepackage{pdflscape}
    \usepackage{afterpage}
    \usepackage{listings,lstautogobble}    
    \usepackage[margin=0.8in]{geometry}
    \usepackage[nottoc,notlot,notlof]{tocbibind}
    \renewcommand\maketitlehookd{\vfill\null}
    \renewcommand\maketitlehooka{\null\mbox{}\vfill}

    \newcommand\backgroundimage{
        \put(-5,0){
        \parbox[b][\paperheight]{\paperwidth}{
        \vfill
        \centering
        \includegraphics[height=\paperheight]{Images/background.jpg}
        \vfill
    }}}

    % Stole code from: https://tex.stackexchange.com/questions/83882/how-to-highlight-python-syntax-in-latex-listings-lstinputlistings-command

    % Default fixed font does not support bold face
    \DeclareFixedFont{\ttb}{T1}{txtt}{bx}{n}{12} % for bold
    \DeclareFixedFont{\ttm}{T1}{txtt}{m}{n}{12}  % for normal

    \definecolor{deepblue}{rgb}{0,0,0.5}
    \definecolor{deepred}{rgb}{0.6,0,0}
    \definecolor{deepgreen}{rgb}{0,0.5,0}

    % SQL style for highlighting
    \lstset{
        language=SQL,
        basicstyle=\small,
        commentstyle=\color{gray},
        otherkeywords={self},
        keywordstyle=\ttb\color{deepblue},
        stringstyle=\color{deepgreen},
        numbers=left,
        numberstyle=\small,
        breaklines=true,
        frame=tb,
        showstringspaces=false,
        autogobble=true
        }

    \graphicspath{ {Images/} }

    \title{EBUS3030 Assignment 1}
    \author{
        Stavros Karmaniolos 
        \texttt{c3160280@uon.edu.au}\\
        Jay Rovacsek
        \texttt{c3146220@uon.edu.au}\\
        Jacob Litherland
        \texttt{c3263482@uon.edu.au}\\
        Edward Lonsdale
        \texttt{c3252144@uon.edu.au}
    }
    \date{\today}
    \hypersetup{
    colorlinks=true,
    linkcolor=black,
    filecolor=magenta,      
    urlcolor=blue,
    citecolor=red,
    linktoc=section,
    }
    \pagenumbering{arabic}

    \newlist{legal}{enumerate}{10}
    \setlist[legal]{label*=\arabic*.}

    \begin{document}
    \AddToShipoutPicture{\backgroundimage}

    \begin{titlingpage}
        \maketitle
    \end{titlingpage}

    \tableofcontents

    \newpage

    \includepdf[pagecommand=\section{Assignment Overview \& Requirements},width=\textwidth,keepaspectratio,pages={1}]{Resources/Assignment1Overview.pdf}
    \includepdf[width=\textwidth,keepaspectratio,pages={2}]{Resources/Assignment1Overview.pdf}

    \subsection{Datamart Business Rules}
    The following business rules were provided to be used in the context of this assignment:
    \begin{itemize}
        \renewcommand\labelitemi{*}
        \item At BIA all customers interacts are in an online environment, 
        there are no orders outside of electronic.
        \item Returning customers can provide POI information via the web
        interface and look up their record and that will flow with the sale.
        \item The sales associate can complete the order form/sale for the
        client.
        \item Each sale will have a receipt number/id.
        \item A receipt can have many line items.
        \item Each line item can only be for a single item, but the customer can
        purchase multiples of the same item.
        \item Where a customer has multiple line items, any sale with more than
        5 row items (containing at least 5 different items) is provided a
        15\% discount.
        \item The system automatically handles the total for the sale by looking
        up the item, then multiplying the costs per item by number
        purchased, and then should store this final field total as a record
        in the system (but should also be able to see clearly sales that
        were provided a discount.
        \item Item prices can change at any point, and the price the customer
        pays is the amount listed for the item on the sale date. We need to
        keep a record of all item prices historically.
        \item Only 1 BIA sales assistant can be attributed to any receipt.
    \end{itemize}

    With these considerations in mind, the following report was created to outline
    the discovery, creation and polish to satisfy the assignment requirements.

    \newpage
    \section{Data Model}
    The below data model is only a suggestion and is still subject to change into the future. A full create script can be found in the \hyperref[sec:Appendix]{\color{blue}appendix}
        \begin{center}
            \includegraphics[height=\textheight-140pt,keepaspectratio]{Images/Suggested_Schema.PNG}
        \end{center}
    It must be noted that the structure of this data model is 
    less than efficient, and it would be expected in a datamart
    situation that only at lower levels of data would this schema
    remain responsive in the manner it is now, as the outline
    suggests the datamart is not necessarily the most suitable
    design for future use, however suits very well currently.
    \par
    It would be expected that only at extremely large data sets
    would this model prove a bad design. In such cases a model 
    more representative of the snowflake or star schema would be
    heavily advised.

    \newpage
    \section{Data Load Process (ETL/ELT)}
        Initial import of the data supplied in the xlsx file generated a very basic table
        that allowed us to analyze the data for potential outliers, confirm the business
        requirements of the data and then create tables from which the data model was derived.
        \\
        The Imported table structure was as follows:
        \begin{center}
            \includegraphics{Images/Initial_Import.PNG}
        \end{center}

        A decision to leave this initial import table as default
        was made to allow easy reference to the initially supplied
        excel data file.
        \par
        In the following sections of \hyperref[sec:QAP]{\color{blue}Quality Assurance Processes}, \hyperref[sec:AR]{\color{blue}Assumptions and Reasoning} and \hyperref[sec:BA]{\color{blue}Base Analysis} we intend to clarify the reasoning behind leaving the imported data in
        the default table suggested by SSMS.

        \newpage

        \subsection{Quality Assurance Processes}
        \label{sec:QAP}
            A number of queries were written to look for data which did not adhere to the spec
            outlined in business requirements and to ensure data was "clean" before entry.
            The first instance of potential issues were encountered with a basic \hyperref[sec:Python]{\color{blue}python} script
            which checked validity of column data, it was found that cells starting at B13777
            to the end of file in the originally supplied excel file were formula values and 
            not static values, this would not have caused an issue with importing into SSMS 
            however certainly broke the script temporarily.
            \par
            After clarifying the issues with the aforementioned cells with Peter, a data file without
            the offending formula was supplied and used for the remainder of the assignment.
            \vspace{5mm}
            \par\noindent
            Discrepancies with some cell formating were noted in Reciept\_Id column in the raw
            data presented to us in the excel file. After testing both an unmodified and a modified
            version of the excel file it was recognised that these cells did not impact the import
            of data into SSMS. The offending cells in question were: B13776 - B13865.
            \\
            The next potential issue encountered was not until a suggested schema structure 
            was complete and data was being scripted to be added to the new schema for analysis.
            The issue encountered was that receipt number 52136 seemed to be an incorrect 
            entry, this was discovered when running the import query for the new schema:

            \begin{lstlisting}
                INSERT INTO Receipt(ReceiptId, ReceiptCustomerId,ReceiptStaffId)
                SELECT DISTINCT(Reciept_Id),Customer_ID,Staff_ID
                FROM Assignment1Data
                ORDER BY Reciept_Id
            \end{lstlisting}

            Which resulted in the error:
            \color{red}
            \begin{Verbatim}[fontsize=\small]
    Violation of PRIMARY KEY constraint 'PK_Receipt'. Cannot insert duplicate key in object 
    'dbo.Receipt'. The duplicate key value is (52136).
            \end{Verbatim}
            \color{black}

            Leading us to recognise that either one of the entries could be incorrect, therefore
            best to investigate both records of the customer Id against the rest of the database:

            \begin{lstlisting}
                SELECT * FROM Assignment1Data 
                WHERE Customer_ID='C32' 
                AND Staff_ID='S15' 
                AND Sale_Date='2017-11-12 00:00:00.0000000';
            
                SELECT * FROM Assignment1Data 
                WHERE Customer_ID='C13' 
                AND Staff_ID='S4' 
                AND Sale_Date='2017-12-30 00:00:00.0000000';
            \end{lstlisting}

            When both queries were performed it was apparent that the data associated with C32 was 
            the likely broken record and modification of the data occurred:

            \begin{lstlisting}
                UPDATE Assignment1Data 
                SET Reciept_Id=51585, 
                Reciept_Transaction_Row_ID=(
                    SELECT MAX(Reciept_Transaction_Row_ID)+1 
                    FROM Assignment1Data
                    WHERE Reciept_Id=51585)
                WHERE Customer_ID='C32' 
                AND Staff_ID='S15' 
                AND Sale_Date='2017-11-12 00:00:00.0000000' 
                AND Item_ID='14';
            \end{lstlisting}

            \newpage

            The next issue arose when again, attempting to run the aforementioned query to import
            into the new Receipt table, this time not one stray record was found, but a complete
            collision on the ReceiptId of 52137, this time as neither record seemed to have 
            records that were correct, it was decided to move one to the maximum ReceiptId + 1:

            \begin{lstlisting}
                UPDATE Assignment1Data 
                SET Reciept_Id=(
                    SELECT MAX(Reciept_Id)+1 
                    FROM Assignment1Data) 
                WHERE Customer_ID='C27' 
                AND Staff_ID='S4' 
                AND Sale_Date='2017-12-30 00:00:00.0000000';
            \end{lstlisting}

            The same issue was replicated on ReceiptId 52138, resolved via:

            \begin{lstlisting}
                UPDATE Assignment1Data 
                SET Reciept_Id=(
                    SELECT MAX(Reciept_Id)+1 
                    FROM Assignment1Data) 
                WHERE Customer_ID='C30' 
                AND Staff_ID='S19' 
                AND Sale_Date='2017-05-16 00:00:00.0000000';
            \end{lstlisting}

            At this point we recognised the broken data likely continued for a while, and 
            evaluated our hypothesis by looking at the original excel file. It turned
            out that data with ReceiptId from 52137-52145 was all broken in the same manner.
            The following query shows this well:

            \begin{lstlisting}
                SELECT Reciept_Id, Customer_ID,Staff_ID 
                FROM Assignment1Data 
                WHERE Reciept_Id BETWEEN 52137 AND 52150
                GROUP BY Reciept_Id, Customer_ID,Staff_ID 
                ORDER BY Reciept_Id;
            \end{lstlisting}

            \newpage

            In order to clean this data we looked at a number of potential methods, with an 
            emphasis on avoiding effort in the task if possible but not breaking the data further,
            which to this point just appeared to be a collision of a number of receipts.
            \\
            We knew a structure such as a CTE\cite{CTE} would allow us to easily split
            distinct records which shared a receiptId and filter by a value such as row number.

            \begin{lstlisting}
            WITH CTE AS
            (
                SELECT ROW_NUMBER() OVER (ORDER BY Reciept_Id) AS RowNumber,
                        Reciept_Id,
                        Customer_ID,
                        Staff_ID
                FROM  Assignment1Data
                WHERE Reciept_Id BETWEEN 52137 AND 52150
                GROUP BY Reciept_Id, Customer_ID,Staff_ID 
            )
            SELECT Reciept_Id,Customer_ID,Staff_ID FROM CTE WHERE (RowNumber % 2 = 0)
            \end{lstlisting}

            Results of the above query yielded:
            \begin{table}[H]
                \centering
                \begin{tabular}{|l|l|l|}
                \hline
                Reciept\_Id & Customer\_Id & Staff\_Id \\ \hline
                52137       & C59          & S2        \\ \hline
                52138       & C30          & S19       \\ \hline
                52139       & C31          & S20       \\ \hline
                52140       & C52          & S10       \\ \hline
                52141       & C42          & S7        \\ \hline
                52142       & C47          & S6        \\ \hline
                52143       & C8           & S13       \\ \hline
                52144       & C50          & S4        \\ \hline
                52145       & C40          & S15       \\ \hline
                52146       & C38          & S5        \\ \hline
                52147       & C9           & S19       \\ \hline
                52148       & C43          & S16       \\ \hline
                52149       & C45          & S11       \\ \hline
                52150       & C57          & S7        \\ \hline
                \end{tabular}
            \end{table}

            \newpage
            
            Whereas the original result without a modulo comparison on the row would have yielded
            a much different result, the raw table supplied in the \hyperref[sec:CTEResults]{\color{blue}appendix}
            \par
            With this known, and additional section was added to the \hyperref[sec:Python]{\color{blue}python} script to generate update statements
            that would be easy to add to the current migrations.sql script we were prototyping.
            \\
            The generated update statements appeared as:

            \begin{lstlisting}
                -- Auto-generated query to fix error of type: Staff.Id Mismatch
                -- Resolved error identified by UUID: dcf16fba08c63ecc85556c385204d9524ec359cf
                UPDATE Assignment1Data 
                SET Reciept_Id=(
                SELECT MAX(Reciept_Id)+1 
                FROM Assignment1Data)
                WHERE Reciept_Id=52136
                AND Customer_Id = 'C13' AND Staff_Id = 'S4'
                GO
            \end{lstlisting}

            Determining now potential entries that broke further rules was our next objective.
            We pursued the idea that entries of receipts could potentially have duplicate items
            recorded against the ReceiptItem table. A simple script was generated to check our 
            assumptions of this:

            \begin{lstlisting}
                -- Verify that no receipt has duplicate ItemIds and all are unique per order
                SELECT *
                FROM
                (
                    SELECT [ReceiptItem].[ReceiptId], 
                    COUNT([ReceiptItem].[ReceiptId]) AS 'ItemCount',
                    COUNT(DISTINCT [ReceiptItem].[ItemId]) AS 'ItemIdCount'
                    FROM [ReceiptItem]
                    GROUP BY [ReceiptItem].[ReceiptId]) AS SubQuery 
                WHERE [SubQuery].[ItemIdCount] != [SubQuery].[ItemCount]
                ORDER BY [SubQuery].[ReceiptId]
                GO
            \end{lstlisting}

            This query returned a result of 912 rows out of the total 2514, which we believed 
            was a large amount given the issues identified earlier numbered in only the teens, 
            however on manual inspection of a number of the reported issue records, it was 
            apparent this figure was actually correct. 
            \par
            Given the large task associated with the entries, an additional module was
            written for generation of SQL in \hyperref[sec:Python]{\color{blue}python} which resulted in two queries for each
            duplicate item entry per receipt, the first query updating the total of one of the 
            records to reflect the real item quantity, the later dropping the non-altered 
            entry after the first had been completed.

            \newpage

            The script was as follows:
            \begin{lstlisting}
                -- Auto-generated query to fix error of type: Item.Id Duplicate
                -- Resolved error identified by UUID: 8b34383524a00eb2097c1c22f870ef2ad104b6b8
                UPDATE Assignment1Data 
                SET [Item_Quantity]=(
                SELECT SUM([Item_Quantity])
                FROM Assignment1Data
                WHERE Reciept_Id=52316
                AND Item_ID = 8)
                WHERE Reciept_Id=52316
                AND Item_ID = 8
                AND Item_Quantity = 10
                GO
                
                -- Auto-generated query to fix error of type: Item.Id Duplicate
                -- Resolved error identified by UUID: 8b34383524a00eb2097c1c22f870ef2ad104b6b8
                DELETE FROM Assignment1Data 
                WHERE Reciept_Id=52316
                AND Item_ID = 8
                AND Item_Quantity < (
                    SELECT MAX([Item_Quantity])
                    FROM Assignment1Data
                    WHERE Reciept_Id=52316
                    AND Item_ID = 8
                )
                GO
            \end{lstlisting}

            Having now cleaned what we believed to be all discrepancies, we could finally start to look at
            evaluating data, our analysis outlined in \hyperref[sec:BA]{\color{blue}base analysis}
        \newpage
        \subsection{Assumptions and Reasoning}
        \label{sec:AR}
            \subsubsection{Item Table}
                An assumption of the ItemId never needing to be larger than a smallint
                was followed, as a basic query into the maximum range within the test data
                suggested that the maximum Id that currently existed was 30:
                \begin{lstlisting}
                    -- Some basic queries for us to determine potential outlier data:
                    -- What is the max of each column where datatype is int?
                    SELECT MAX(Item_ID) AS 'Max Item_ID'
                    FROM Assignment1Data;
                \end{lstlisting}

                With the results:

                ItemDescription underwent some size optimisation, as the max data length 
                that currently existed within the supplied data was 52, and we are to assume
                that into the future more items may be added, a value of 255 should allow
                for a varied range of descriptions.
                \\
                SQL queried to determine to above assumption:

                \begin{lstlisting}
                    -- Determine current max varchar used in Item_Description
                    SELECT MAX(DATALENGTH(Item_Description)) 
                    FROM Assignment1Data;
                \end{lstlisting}

                We do recognise the requirements for optimisation may not require such measures, and 
                acknowledge that a varchar(max)/text datatype would also be reasonable.
            \subsubsection{Price Table}
                The price table was designed to hold historical data as required by the business rules,
                an effective range can be used here to determine item pricing for time frames,
                current items having no end date or an end date as some point in time into the future.
                \par
                Accuracy on the pricing was important, we decided to use a decimal(19,5) structure to
                ensure no problems should arise at any point with calculation of totals.\cite{MoneyIssues}
                \\
                Another notable feature of the price table is the relationships with both
                item and receiptItem, which allows the receiptItem table to point at a price value 
                that can either be current or historical in nature.
            \subsubsection{ReceiptItem}
                The receipt item table acts as a line-item style associative entity, the quantity and 
                historical priceId used at time of transaction can allow an item's price to be
                updated and still maintain historical pricing associated with the receipt.
                \par
                As noted above, to facilitate historical value lookup, this table also holds relationships
                with the price table.
            \subsubsection{Receipt}
                The receipt table acts as a meta-table in this instance, other tables associate with 
                this table with the receiptId field. Due to this it made it extremely easy to use a number
                of joins/inner joins to determine some of the metrics outlined in the \hyperref[sec:BA]{base analysis}.
            \subsubsection{Staff}
                Staff was left in a non-normalised state to ensure efficiency of queries into the future, 
                normalising the table further would yield little value to the business based on the requirements.
                The office table is referenced by the staff table. This is merely to satisfy the assumption
                that, while the only office to exist was Newcastle in this setting, the requirement of more 
                offices into the future is a possibility and the required join would be little impact on speed
                of queries in a datamart.
            \subsubsection{Customer}
                Customer, just like staff could be normalised further requiring more joins and potentially 
                causing a performance issue into the future, for simplicity we kept only the supplied
                data in mind, and assumed no more data would be required by the datamart into the future.

    \section{Base Analysis}
    \label{sec:BA}
        \subsection{Raw Results}
            A number of metrics were considered to satisfy the request related to the best salesperson,
            as we are not certain if this is determined by a specific metric or a set of metrics we 
            included a number of analyzed points for the project:
            \begin{itemize}
                \item Total receipts attributed to a staff member
                \item Total items sold by a staff member
                \item Ratio of discounted sales to normal sales for each staff member
                \item Total sale value per staff member
                \item Average sale value per staff member
                \item Average item value per staff member
            \end{itemize}

            \subsubsection{Total Number of Sales}
                The total number of sales per staff member were considered with the following 
                sql query:
                \begin{lstlisting}
                    -- Sales count per staff member (Receipt Count)
                    SELECT COUNT(*) AS 'Sales Count', s.StaffId,s.StaffFirstName,s.StaffSurname
                    FROM Receipt r
                    INNER JOIN ReceiptItem ri ON r.ReceiptId = ri.ReceiptId
                    INNER JOIN Item i ON i.ItemId = ri.ItemId
                    INNER JOIN Price p ON p.PriceId = ri.PriceId
                    INNER JOIN Staff s ON s.StaffId = r.ReceiptStaffId
                    GROUP BY s.StaffId,s.StaffFirstName,s.StaffSurname
                    ORDER BY 'Sales Count' DESC;
                \end{lstlisting}

                Leading to a range of 700 to 478, the top five staff were:

                \begin{table}[H]
                    \centering
                    \begin{tabular}{|l|l|l|l|}
                    \hline
                    Sales Count & StaffId & StaffFirstName & StaffSurname \\ \hline
                    700         & S17     & Daniel         & Baker        \\ \hline
                    682         & S19     & Kaitlyn        & Ortiz        \\ \hline
                    676         & S8      & Michelle       & Miller       \\ \hline
                    664         & S5      & Stephanie      & Watson       \\ \hline
                    664         & S6      & Evan           & Hill         \\ \hline
                    \end{tabular}
                    \end{table}

            \subsubsection{Total Items Sold}
                The total items attributed to each staff member were considered also,
                determined by the query:
                
                \begin{lstlisting}
                    -- Item count per staff member
                    SELECT SUM(ri.ReceiptItemQuantity) AS 'Item Count', s.StaffId,s.StaffFirstName,s.StaffSurname
                    FROM Receipt r
                    INNER JOIN ReceiptItem ri ON r.ReceiptId = ri.ReceiptId
                    INNER JOIN Staff s ON s.StaffId = r.ReceiptStaffId
                    GROUP BY s.StaffId,s.StaffFirstName,s.StaffSurname
                    ORDER BY 'Item Count' DESC;
                \end{lstlisting}

                Yielding a range of 4217 to 2813, with the top five staff members in this
                analysis:
                \begin{table}[H]
                    \centering
                    \begin{tabular}{|l|l|l|l|}
                    \hline
                    Item Count & StaffId & StaffFirstName & StaffSurname \\ \hline
                    4217       & S19     & Kaitlyn        & Ortiz        \\ \hline
                    4212       & S17     & Daniel         & Baker        \\ \hline
                    4144       & S8      & Michelle       & Miller       \\ \hline
                    4052       & S1      & Lauren         & Martin       \\ \hline
                    4036       & S5      & Stephanie      & Watson       \\ \hline
                    \end{tabular}
                \end{table}
            \subsubsection{Discounted Sales Ratio}
            Consideration of the number of sales made by each staff member was also made,
            the following query yielding the results we required:

            \begin{lstlisting}
                -- Sales metrics for discounted and standard sales per staff member
                SELECT s.StaffId,s.StaffFirstName,s.StaffSurname,
                SUM(SubQuery.[Discounted Sales]) AS 'Discounted Sales',
                SUM(SubQuery.[Standard Sales]) AS 'Standard Sales'
                FROM (
                    SELECT CAST(
                        CASE
                        WHEN COUNT(ri.[ReceiptItemQuantity]) >= 5
                            THEN 1
                        ELSE 0
                        END AS int) AS 'Discounted Sales',
                    CAST(
                        CASE
                        WHEN COUNT(ri.[ReceiptItemQuantity]) >= 5
                            THEN 0
                        ELSE 1
                    END AS int) AS 'Standard Sales',
                    r.ReceiptId
                    FROM Receipt r
                    INNER JOIN ReceiptItem ri ON r.ReceiptId = ri.ReceiptId
                    INNER JOIN Item i ON i.ItemId = ri.ItemId
                    INNER JOIN Price p ON p.PriceId = ri.PriceId
                    GROUP BY r.ReceiptId
                ) AS SubQuery
                INNER JOIN Receipt r ON SubQuery.ReceiptId = r.ReceiptId
                INNER JOIN ReceiptItem ri ON r.ReceiptId = ri.ReceiptId
                INNER JOIN Staff s ON s.StaffId = r.ReceiptStaffId
                GROUP BY s.StaffId,s.StaffFirstName,s.StaffSurname
            \end{lstlisting}

            \newpage
            Results from the query yielded:

            \begin{table}[H]
                \centering
                \begin{tabular}{|l|l|l|l|l|l|}
                \hline
                StaffId & StaffFirstName & StaffSurname & Discounted Sales & Standard Sales & Discount Rate (\%) \\ \hline
                S4      & Robert         & Wood         & 518              & 115            & 81.83\%            \\ \hline
                S14     & Noah           & Brooks       & 533              & 119            & 81.75\%            \\ \hline
                S1      & Lauren         & Martin       & 533              & 126            & 80.88\%            \\ \hline
                S16     & Jordan         & Turner       & 520              & 123            & 80.87\%            \\ \hline
                S15     & Bailey         & Green        & 500              & 124            & 80.13\%            \\ \hline
                S13     & Molly          & Carter       & 527              & 131            & 80.09\%            \\ \hline
                S17     & Daniel         & Baker        & 556              & 144            & 79.43\%            \\ \hline
                S20     & Dylan          & Hall         & 505              & 132            & 79.28\%            \\ \hline
                S6      & Evan           & Hill         & 524              & 140            & 78.92\%            \\ \hline
                S10     & Jonathan       & Jenkins      & 454              & 123            & 78.68\%            \\ \hline
                S5      & Stephanie      & Watson       & 520              & 144            & 78.31\%            \\ \hline
                S18     & Megan          & James        & 508              & 142            & 78.15\%            \\ \hline
                S19     & Kaitlyn        & Ortiz        & 531              & 151            & 77.86\%            \\ \hline
                S7      & Molly          & Jackson      & 474              & 141            & 77.07\%            \\ \hline
                S9      & Mélissa        & Garcia       & 489              & 147            & 76.89\%            \\ \hline
                S8      & Michelle       & Miller       & 509              & 167            & 75.30\%            \\ \hline
                S12     & Leah           & Harris       & 356              & 122            & 74.48\%            \\ \hline
                S11     & Gavin          & Thompson     & 395              & 137            & 74.25\%            \\ \hline
                S2      & Joseph         & Reed         & 447              & 160            & 73.64\%            \\ \hline
                S3      & Amber          & Hill         & 396              & 168            & 70.21\%            \\ \hline
                \end{tabular}
                \end{table}

            \newpage
            \subsubsection{Total Sales Value per Staff Member}

            Consideration of the total sales per staff member was considered a highly important 
            metric to consider also, we did consider comparing the results of this to the 
            results of a query that did not include discount to see whom would be considered
            the best performer if discounts were not relevant, however we also recognise this to be too
            speclutive in nature. The required query was as follows:

            \begin{lstlisting}
            -- Sales total per staff with discounts applied ($)
            SELECT CAST(
                    CASE
                    WHEN COUNT(ri.[ReceiptItemQuantity]) >= 5
                        THEN SUM(p.[Price] * ri.[ReceiptItemQuantity]) * 0.85
                    ELSE SUM(p.[Price] * ri.[ReceiptItemQuantity])
                    END AS decimal(19,5)) AS 'Sales Totals',
                    s.StaffId,s.StaffFirstName,s.StaffSurname
            FROM Receipt r
            INNER JOIN ReceiptItem ri ON r.ReceiptId = ri.ReceiptId
            INNER JOIN Item i ON i.ItemId = ri.ItemId
            INNER JOIN Price p ON p.PriceId = ri.PriceId
            INNER JOIN Staff s ON s.StaffId = r.ReceiptStaffId
            INNER JOIN Customer c ON c.CustomerId = r.ReceiptCustomerId
            GROUP BY s.StaffId,s.StaffFirstName,s.StaffSurname
            ORDER BY 'Sales Totals' DESC;
            \end{lstlisting}

            Resulting in the following results:

            \begin{table}[H]
                \centering
                \begin{tabular}{|l|l|l|l|}
                \hline
                Sales Totals & StaffId & StaffFirstName & StaffSurname \\ \hline
                78572.21500  & S8      & Michelle       & Miller       \\ \hline
                73847.10750  & S19     & Kaitlyn        & Ortiz        \\ \hline
                72764.88750  & S17     & Daniel         & Baker        \\ \hline
                71699.66750  & S14     & Noah           & Brooks       \\ \hline
                70514.68250  & S3      & Amber          & Hill         \\ \hline
                69182.56250  & S2      & Joseph         & Reed         \\ \hline
                69051.19500  & S13     & Molly          & Carter       \\ \hline
                68831.81000  & S5      & Stephanie      & Watson       \\ \hline
                68133.83250  & S4      & Robert         & Wood         \\ \hline
                66267.19000  & S6      & Evan           & Hill         \\ \hline
                65018.37000  & S10     & Jonathan       & Jenkins      \\ \hline
                64002.06750  & S1      & Lauren         & Martin       \\ \hline
                63760.66750  & S16     & Jordan         & Turner       \\ \hline
                62304.70250  & S18     & Megan          & James        \\ \hline
                61862.44750  & S9      & Mélissa        & Garcia       \\ \hline
                61832.27250  & S15     & Bailey         & Green        \\ \hline
                61536.85500  & S20     & Dylan          & Hall         \\ \hline
                58996.16250  & S7      & Molly          & Jackson      \\ \hline
                52102.66250  & S11     & Gavin          & Thompson     \\ \hline
                50259.35250  & S12     & Leah           & Harris       \\ \hline
                \end{tabular}
            \end{table}

            \newpage

            \subsubsection{Average Value Per Sale}

            The average receipt value per staff member was another metric we considered would add
            value to the descision to be suggested in the 
            \hyperref[sec:Executive Summary]{\color{blue}executive summary}.
            The required query to determine this metric was as follows:

            \begin{lstlisting}
            -- Sales average per staff with discounts applied
            SELECT (CAST(
                    CASE
                    WHEN COUNT(ri.[ReceiptItemQuantity]) >= 5
                        THEN SUM(p.[Price] * ri.[ReceiptItemQuantity]) * 0.85
                    ELSE SUM(p.[Price] * ri.[ReceiptItemQuantity])
                    END AS decimal(19,5)) / COUNT(r.ReceiptId)) AS 'Sales Average',
                    s.StaffId,s.StaffFirstName,s.StaffSurname
            FROM Receipt r
            INNER JOIN ReceiptItem ri ON r.ReceiptId = ri.ReceiptId
            INNER JOIN Item i ON i.ItemId = ri.ItemId
            INNER JOIN Price p ON p.PriceId = ri.PriceId
            INNER JOIN Staff s ON s.StaffId = r.ReceiptStaffId
            GROUP BY s.StaffId,s.StaffFirstName,s.StaffSurname
            ORDER BY 'Sales Average' DESC;
            \end{lstlisting}

            With results as follows:

            \begin{table}[H]
                \centering
                \begin{tabular}{|l|l|l|l|}
                \hline
                Sales Average        & StaffId & StaffFirstName & StaffSurname \\ \hline
                125.0260328014184397 & S3      & Amber          & Hill         \\ \hline
                116.2310872781065088 & S8      & Michelle       & Miller       \\ \hline
                113.9745675453047775 & S2      & Joseph         & Reed         \\ \hline
                112.6834835355285961 & S10     & Jonathan       & Jenkins      \\ \hline
                109.9688151840490797 & S14     & Noah           & Brooks       \\ \hline
                108.2802162756598240 & S19     & Kaitlyn        & Ortiz        \\ \hline
                107.6363862559241706 & S4      & Robert         & Wood         \\ \hline
                105.1450889121338912 & S12     & Leah           & Harris       \\ \hline
                104.9410258358662613 & S13     & Molly          & Carter       \\ \hline
                103.9498392857142857 & S17     & Daniel         & Baker        \\ \hline
                103.6623644578313253 & S5      & Stephanie      & Watson       \\ \hline
                99.7999849397590361  & S6      & Evan           & Hill         \\ \hline
                99.1612247278382581  & S16     & Jordan         & Turner       \\ \hline
                99.0901802884615384  & S15     & Bailey         & Green        \\ \hline
                97.9373355263157894  & S11     & Gavin          & Thompson     \\ \hline
                97.2679992138364779  & S9      & Mélissa        & Garcia       \\ \hline
                97.1199810318664643  & S1      & Lauren         & Martin       \\ \hline
                96.6041679748822605  & S20     & Dylan          & Hall         \\ \hline
                95.9287195121951219  & S7      & Molly          & Jackson      \\ \hline
                95.8533884615384615  & S18     & Megan          & James        \\ \hline
                \end{tabular}
            \end{table}

            We consider this to be a metric which weighs heavily in our analysis,
            as multiple factors would impact this result, the number of items on the sale
            (resulting in a lower total if discount was applied). Another consideration 
            for this metric would be that it leans towards anyone who could
            sell a larger quantity of the same item, as this lends itself towards a higher
            receipt total. 
			
			\par
			We see that Ms Amber Hill (S3) has the highest average sale total and this is backed up by
			her high item sales count, indicating she is making more sales per receipt on average than
			any other sales office. 
			However, Ms Hill has not made as much revenue as some other employees, approximately \$7,500
			behind the sales leader who achieved \$78,572. Originally, we suspected that perhaps Ms Hill
			was a new employee but reviewing sale receipt dates we can confirm this was not a valid
			assumption and that she was working within the business throughout the entirety of 2017.
			After this revelation, we can rule Ms Hill out of our assessment for the best sales person.
			The data suggests that Ms Hill sells higher cost items which makes up for the lack of
			transactions she completes when compared to other staff members. 
			
        \subsection{Notes on Analysis}
			As a group, we emphasised identifying and avoiding bad data as our top priority for
			the analysis outlined in this report. One of the major steps taken in this process 
			included
			the removal of duplicate items on receipts and consolidating them into single line
			items. The reasoning behind this is that we discovered a number of receipts that
			included redundant entries which showed double or, in rare cases, triple the number
			of items expected which would falsely inflate the cost of items on a receipt.

    \newpage
    
    
    
    \section{Executive Summary}
    \label{sec:Executive Summary}
	This report was created for the head sales executives of BIA Inc for the purpose of
	determining which sales staff member would be considered the best performer based on
	the data set provede by the firm. As no specific metrics or measurement requirements
	were provided, an analysis was performed by our team. This analysis included extracting,
	cleaning and loading the data using SQL scripts and a supplementary Python script which
	assisted in preparing the data provided by the firm for analysis and query in SQL Server
	Management Studio (SSMS). 
    \vspace{5mm}
    \par\noindent
    
    \noindent The findings of our analysis resulted in ranking high achieving staff members
    determined by a number of key metrics selected by the team. Firstly, we ranked the sales
    officers by total number of sales and, from this group, identified the top staff member
    of sales, Mr Daniel Baker. Mr Baker had made the largest number of sales in the 12 months
    of data supplied with a total of 700 sales. Placed immediately after Mr Baker, with 18 fewer
    sales is Ms Kaitlyn Ortiz (682), then Ms Michelle Miller (676), followed by Ms Stephanie
    Watson (664) and Mr Evan Hill (664).
    \vspace{5mm}
    \par\noindent
    
    \noindent The next metric was total items sold. In this relation, the best sales officer
    is Ms Kaitlyn Ortiz, the second place sales officer in the first metric. In the 12 months
    of data supplied, it can be observed that of the 682 total sales, Ms Ortiz sold 4217 items
    with approximately six (6.18) items per sale. The second place sales officer, Mr Daniel
    Baker, sold 4212 items, only five items less than Ms Ortiz, and also sold approximately
    six (6.02) items per sale. The third placed employee in this relation is Ms Michelle Miller
    who resulted in 4414 total sales and approximately six (6.13) items per sale.
    \vspace{5mm}
    \par\noindent

    \noindent The third key metric is discounted sales ratio. As stated in the business rules
    document provided by the firm, any sale with five or more row items would be eligible for
    a 15\% discount to the total sale. We identified this as an important factor because
    understanding how many items were discounted may offer insight into sales methods and
    techniques applied by sales officers which can then be used to enhance future performance.
    From the results, we aggregated the data by total percentage of sales which were discounted.
    Mr Robert Wood (84.57\%) discounted the greatest share of his sales of all sales staff
    members. After Mr Wood comes Mr Dylan Hall (83.41\%), then Ms Lauren Martin (83.31\%), Mr 
    Jordan Turner (82.74\%), Mr Noah Brooks (82.72\%) and Mr Daniel Baker (81.39\%).
    \vspace{5mm}
    \par
    
	\noindent The final metric considered was total sales value per staff member. The purpose
	of this report is to find the ``best'' salesperson, therefore, we can assume that, along
	with other metrics, the firm would be interested to know which sales officer generates
	the most revenue. After examining the data, we can state that Ms Michelle Miller is the sales
	officer that generates the most value with a total of \$78,572.22 over 12 months. This
	equates to \$4,725.10 more than her nearest competitor, Mr Daniel Baker who generated
	\$72,764.89 who was followed by Mr Noah Brooks with \$71,699.67 and Ms Amber Hill with 
	\$70,514.68.
	\vspace{5mm}    
	\par
    

    \noindent After carefully considering the aforementioned key metrics and reviewing the
    results, we can conclude that Ms Michelle Miller should be considered the most valuable
    sales office at BIA Inc. Our findings showed clearly that Ms Michelle Miller achieved
    the highest sales value when compared with other sales officer by a significant margin 
    (\$4,725.10). While Ms Miller was not ranked first in total number of sales or total items
    sold, she did rank highly in both relations. Ms Miller did however score poorly in her
    discounted sales ratio. This could indicate that Ms Miller is not as effective at
    `upselling' as her fellow sales officers and this could be an area for improvement.
    We can assert that Ms Miller should be considered for the reward (and possible cash
    prize) suggested in the original document outlining the firm requirements. If for any
    reason Ms Miller should not be applicable or eligible for the discount, we would recommend
    Ms Kaitlyn Ortiz as the alternative choice due to her high ranking in all metrics discussed
    in this summary.
     

    \newpage
    \begin{thebibliography}{9}
        \raggedright
        \bibitem{MoneyIssues}
            Reasons against TSQL Money type: Stackoverflow User; \textit{SQLMenace}
            \url{https://stackoverflow.com/questions/582797/should-you-choose-the-money-or-decimalx-y-datatypes-in-sql-server}
        \bibitem{Numeric}
            Microsoft TSQL documentation of Decimal/Numeric types
            \url{https://docs.microsoft.com/en-us/sql/t-sql/data-types/decimal-and-numeric-transact-sql?view=sql-server-2017}
        \bibitem{CTE}
        Microsoft documentation: WITH common\_table\_expression (Transact-SQL)
            \url{https://docs.microsoft.com/en-us/sql/t-sql/queries/with-common-table-expression-transact-sql?view=sql-server-2017}
        \bibitem{BusDictionairyUpselling}
        		Upselling - Business Dictionary
        		\url{http://www.businessdictionary.com/definition/upselling.html}
    \end{thebibliography}

    \newpage
    \section{Appendix}
    \label{sec:Appendix}
    \subsection{CTE Raw Results}
    \label{sec:CTEResults}
    \begin{table}[H]
        \centering
        \begin{tabular}{|l|l|l|}
        \hline
        Reciept\_Id & Customer\_Id & Staff\_Id \\ \hline
        52137       & C27          & S4        \\ \hline
        52137       & C59          & S2        \\ \hline
        52138       & C29          & S13       \\ \hline
        52138       & C30          & S19       \\ \hline
        52139       & C3           & S5        \\ \hline
        52139       & C31          & S20       \\ \hline
        52140       & C38          & S4        \\ \hline
        52140       & C52          & S10       \\ \hline
        52141       & C24          & S19       \\ \hline
        52141       & C42          & S7        \\ \hline
        52142       & C46          & S8        \\ \hline
        52142       & C47          & S6        \\ \hline
        52143       & C51          & S17       \\ \hline
        52143       & C8           & S13       \\ \hline
        52144       & C11          & S10       \\ \hline
        52144       & C50          & S4        \\ \hline
        52145       & C21          & S8        \\ \hline
        52145       & C40          & S15       \\ \hline
        52146       & C38          & S16       \\ \hline
        52146       & C38          & S5        \\ \hline
        52147       & C40          & S18       \\ \hline
        52147       & C9           & S19       \\ \hline
        52148       & C26          & S8        \\ \hline
        52148       & C43          & S16       \\ \hline
        52149       & C10          & S19       \\ \hline
        52149       & C45          & S11       \\ \hline
        52150       & C15          & S10       \\ \hline
        52150       & C57          & S7        \\ \hline
        \end{tabular}
    \end{table}

    \newpage
    \subsection{Python Script}
    \label{sec:Python}

    % Python style for highlighting
    \newcommand\pythonstyle{\lstset{
        language=Python,
        basicstyle=\ttm,
        otherkeywords={self},
        keywordstyle=\ttb\color{deepblue},
        emph={
            Item,
            Office,
            Staff,
            Customer,
            Receipt,
            Sale,
            Sales,
            Employees,
            Items,
            Error_Log,
            Customers,
            LoggedErrors,
            __init__,
            },
        emphstyle=\ttb\color{deepred},
        stringstyle=\color{deepgreen},
        breaklines=true,
        frame=tb,
        showstringspaces=false
        }}
    
        % Python environment
        \lstnewenvironment{python}[1][]
        {
        \pythonstyle
        \lstset{#1}
        }
        {}
    
        % Python for external files
        \newcommand\pythonexternal[2][]{{
            \pythonstyle
            \lstinputlisting[#1]{#2}}}

    \begin{center}
        \pythonexternal{Data_Analysis/Scripts/parsedata/main.py}
    \end{center}

    \newpage

    \begin{center}
        \pythonexternal{Data_Analysis/Scripts/parsedata/classes.py}
    \end{center}
    
    \end{document}