\documentclass{article}
    \usepackage{url}
    \usepackage{cite}
    \usepackage{float}   
    \usepackage{xcolor}
    \usepackage{lscape}
    \usepackage{amssymb}
    \usepackage{titling}
    \usepackage{pdfpages}
    \usepackage{enumitem}
    \usepackage{graphicx}
    \usepackage{hyperref}
    \usepackage{fancybox}
    \usepackage{fancyvrb}
    \usepackage{enumerate}
    \usepackage{pdflscape}
    \usepackage{afterpage}
    \usepackage{listings,lstautogobble}    
    \usepackage[margin=0.8in]{geometry}
    \usepackage[nottoc,notlot,notlof]{tocbibind}
    \renewcommand\maketitlehookd{\vfill\null}
    \renewcommand\maketitlehooka{\null\mbox{}\vfill}

    \newcommand\backgroundimage{
        \put(-5,0){
        \parbox[b][\paperheight]{\paperwidth}{
        \vfill
        \centering
        %\includegraphics[height=\paperheight]{Images/background.jpg}
        \vfill
    }}}

    % Stole code from: https://tex.stackexchange.com/questions/83882/how-to-highlight-python-syntax-in-latex-listings-lstinputlistings-command

    % Default fixed font does not support bold face
    \DeclareFixedFont{\ttb}{T1}{txtt}{bx}{n}{12} % for bold
    \DeclareFixedFont{\ttm}{T1}{txtt}{m}{n}{12}  % for normal

    \definecolor{deepblue}{rgb}{0,0,0.5}
    \definecolor{deepred}{rgb}{0.6,0,0}
    \definecolor{deepgreen}{rgb}{0,0.5,0}

    % SQL style for highlighting
    \lstset{
        language=SQL,
        basicstyle=\small,
        commentstyle=\color{gray},
        otherkeywords={self},
        keywordstyle=\ttb\color{deepblue},
        stringstyle=\color{deepgreen},
        numbers=left,
        numberstyle=\small,
        breaklines=true,
        frame=tb,
        showstringspaces=false,
        autogobble=true
        }

    \graphicspath{ {Images/} }

    \title{EBUS3030 Assignment 2}
    \author{
        Stavros Karmaniolos 
        \texttt{c3160280@uon.edu.au}\\
        Jay Rovacsek
        \texttt{c3146220@uon.edu.au}\\
        Jacob Litherland
        \texttt{c3263482@uon.edu.au}\\
        Edward Lonsdale
        \texttt{c3252144@uon.edu.au}
    }
    \date{\today}
    \hypersetup{
    colorlinks=true,
    linkcolor=black,
    filecolor=magenta,      
    urlcolor=blue,
    citecolor=red,
    linktoc=section,
    }
    \pagenumbering{arabic}

    \newlist{legal}{enumerate}{10}
    \setlist[legal]{label*=\arabic*.}

    \begin{document}
    \AddToShipoutPicture{\backgroundimage}

    \begin{titlingpage}
        \maketitle
    \end{titlingpage}

    \tableofcontents

    \newpage
    
% ------------------------------------------------------------------------------------------------ %
% ASSIGNMENT OUTLINE
% ------------------------------------------------------------------------------------------------ %

    \includepdf[pagecommand=\section{Assignment Overview \& Requirements},width=\textwidth,keepaspectratio,pages={1}]{Resources/Assignment_2_Overview.pdf}
    \includepdf[width=\textwidth,keepaspectratio,pages={2}]{Resources/Assignment_2_Overview.pdf}

% ------------------------------------------------------------------------------------------------ %
% EXECUTIVE SUMMARY
% ------------------------------------------------------------------------------------------------ %

    \section{Executive Summary}
    \label{sec:Executive Summary}
	
    \newpage


% ------------------------------------------------------------------------------------------------ %
% BUSINESS RULES
% ------------------------------------------------------------------------------------------------ %

    \subsection{Datamart Business Rules}
    The following business rules were provided to be used in the context of this assignment:
    \begin{itemize}
        
        \item At BIA all customers interacts are in an online environment, all orders are electronic.
        \item Returning customers can provide POI information via the web interface and look up their
        record and that will flow with the sale.
        \item The sales associate can complete the order form/sale for the client.
        \item Each sale will have a receipt number/id.
        \item A receipt can have many line items.
        \item Each line item can only be for a single item, but the customer can purchase multiples 
        of the same item.
        \item Where a customer has multiple line items, any sale with 5 or more row items 
        (containing at least five (5) different items) is provided a 5\% discount.
        \item The system automatically handles the total for the sale by looking up the item, then 
        multiplying the costs per item by number purchased, and then should store this final field 
        total as a record in the system (but should also be able to see clearly sales that were 
        provided a discount. 
        \item Item prices can change at any point, and the price the customer
        pays is the amount listed for the item on the sale date. We need to
        keep a record of all item prices historically so that we can determine what the store item 
        price was at any particular past date.
        \item Only one (1) BIA sales assistant can be attributed to any receipt.
        \item Customers may visit multiple stores for purchases (ie they are not locked to a particular 
        store). As a result, all customer records are replicated across all stores, so they do not need 
        to be re-recorded at a store by store level.
    \end{itemize}
   
    With these considerations in mind, the following report was created to outline
    the discovery, creation and polish to satisfy the assignment requirements.




% ------------------------------------------------------------------------------------------------ %
% DATA MODEL
% ------------------------------------------------------------------------------------------------ %

    \newpage
    \section{Data Model}
    The below data model is only a suggestion and is still subject to change into the future. A full create script can be found in the \hyperref[sec:Appendix]{\color{blue}appendix}
        \begin{center}
            \includegraphics[width=\textwidth,keepaspectratio]{Images/schema.PNG}
        \end{center}
    It must be noted that the structure of this data model is 
    less than efficient, and it would be expected in a datamart
    situation that only at lower levels of data would this schema
    remain responsive in the manner it is now, as the outline
    suggests the datamart is not necessarily the most suitable
    design for future use, however suits very well currently.
    \par
    It would be expected that only at extremely large data sets
    would this model prove a bad design. In such cases a model 
    more representative of the snowflake or star schema would be
    heavily advised.

    \newpage
    An EER diagram of the suggested data model:
    \begin{center}
        %\includegraphics[width=\textwidth-40pt,keepaspectratio]{Images/EER_diagram.png}
    \end{center}



% ------------------------------------------------------------------------------------------------ %
% DATA LOAD PROCESS
% ------------------------------------------------------------------------------------------------ %

    \newpage
    \section{Data Load Process (ETL/ELT)}
        Initial import of the data supplied in the xlsx file generated a very basic table
        that allowed us to analyze the data for potential outliers, confirm the business
        requirements of the data and then create tables from which the data model was derived.
        \\
        The Imported table structure was as follows:
        \begin{center}
            %\includegraphics{Images/Initial_Import.PNG}
        \end{center}

        \noindent
        A decision to leave this initial import table as default
        was made to allow easy reference to the initially supplied
        excel data file.
        % \par
        % In the following sections of \hyperref[sec:QAP]{\color{blue}Quality Assurance Processes}, \hyperref[sec:AR]{\color{blue}Assumptions and Reasoning} and \hyperref[sec:BA]{\color{blue}Base Analysis} we intend to clarify the reasoning behind leaving the imported data in
        % the default table suggested by SSMS.

        \newpage



% ------------------------------------------------------------------------------------------------ %
% QA PROCESS
% ------------------------------------------------------------------------------------------------ %

        \subsection{Quality Assurance Processes}
        \label{sec:QAP}
            Maybe include some C\# code references or whatnot.
% ------------------------------------------------------------------------------------------------ %
% ASSUMPTIONS/REASONING
% ------------------------------------------------------------------------------------------------ %

        \newpage
        \subsection{Assumptions and Reasoning}
        \label{sec:AR}
            \subsubsection{Item}

            \subsubsection{ReceiptItem}

            \subsubsection{Receipt}

            \subsubsection{Staff}

            \subsubsection{Customer}

            \subsubsection{Office}

% ------------------------------------------------------------------------------------------------ %
% BASE ANALYSIS
% ------------------------------------------------------------------------------------------------ %
    \section{Base Analysis}
    \label{sec:BA}

        \subsection{Notes on Analysis}


            \subsection{Raw Results}

            % A number of metrics were considered to satisfy the request related to the best salesperson,
            % as we are not certain if this is determined by a specific metric or a set of metrics we 
            % included a number of analyzed points for the project:
            % \begin{itemize}
            %     \item Total receipts attributed to a staff member
            %     \item Total items sold by a staff member
            %     \item Ratio of discounted sales to normal sales for each staff member
            %     \item Total sale value per staff member
            %     \item Average sale value per staff member
            %     \item Average item value per staff member
            % \end{itemize}

            \subsubsection{Total Number of Sales}
                The total number of sales per staff member were considered with the following 
                sql query:
                \begin{lstlisting}
                    -- Sales count per staff member (Receipt Count)
                    SELECT COUNT(*) AS 'Sales Count', s.StaffId,s.StaffFirstName,s.StaffSurname, o.OfficeId, o.OfficeLocation
                    FROM Receipt r
                    INNER JOIN ReceiptItem ri ON r.ReceiptId = ri.ReceiptId
                    INNER JOIN Item i ON i.ItemId = ri.ItemId
                    INNER JOIN Staff s ON s.StaffId = r.ReceiptStaffId
                    INNER JOIN Office o  ON o.OfficeId = s.StaffOfficeId
                    GROUP BY s.StaffId,s.StaffFirstName,s.StaffSurname, o.OfficeId, o.OfficeLocation 
                    ORDER BY 'Sales Count' DESC;
                \end{lstlisting}

                \begin{table}[H]
                    \centering
                    \begin{tabular}{|l|l|l|l|}
                    \hline
                    Sales Count & StaffId & StaffFirstName & StaffSurname & OfficeId & OfficeLocation \\ \hline
                    720         & S190    & Samuel         & Anderson     & 4        & Sydney         \\ \hline
                    688         & S122    & Austin         & Morris       & 6        & Grafton        \\ \hline
                    678         & S196    & Devin          & Brown        & 6        & Grafton        \\ \hline
                    666         & S45     & Emma           & Gutierrez    & 3        & Cessnock       \\ \hline
                    658         & S101    & Jenna          & Cox          & 5        & Port Macquarie \\ \hline
                    \end{tabular}
                    \end{table}

            \subsubsection{Total Items Sold}
                The total items attributed to each staff member were considered also,
                determined by the query:
                
                \begin{lstlisting}
                    -- Item count per staff member
                    SELECT SUM(ri.ReceiptItemQuantity) AS 'Item Count', s.StaffId,s.StaffFirstName,s.StaffSurname, o.OfficeId, o.OfficeLocation
                    FROM Receipt r
                    INNER JOIN ReceiptItem ri ON r.ReceiptId = ri.ReceiptId
                    INNER JOIN Staff s ON s.StaffId = r.ReceiptStaffId
                    INNER JOIN Office o  ON o.OfficeId = s.StaffOfficeId
                    GROUP BY s.StaffId,s.StaffFirstName,s.StaffSurname, o.OfficeId, o.OfficeLocation
                    ORDER BY 'Item Count' DESC;
                \end{lstlisting}

                % Yielding a range of 4217 to 2813, with the top five staff members in this
                % analysis:

                \begin{table}[H]
                    \centering
                    \begin{tabular}{|l|l|l|l|}
                    \hline
                    Item Count & StaffId & StaffFirstName & StaffSurname & OfficeId & OfficeLocation \\ \hline
                    3978       & S190    & Samuel         & Anderson     & 4        & Sydney         \\ \hline
                    3787       & S122    & Austin         & Morris       & 6        & Grafton        \\ \hline
                    3683       & S45     & Emma           & Gutierrez    & 3        & Cessnock       \\ \hline
                    3679       & S101    & Jenna          & Cox          & 5        & Port Macquarie \\ \hline
                    3628       & S106    & Mia            & Foster       & 9        & Wagga Wagga    \\ \hline
                    \end{tabular}
                    \end{table}

            \subsubsection{Discounted Sales Ratio}
                Consideration of the number of sales made by each staff member was also made,
                the following query yielding the results we required:

            \begin{lstlisting}
                -- Sales metrics for discounted and standard sales per staff member
                SELECT s.StaffId,s.StaffFirstName,s.StaffSurname, o.OfficeId, o.OfficeLocation
                SUM(SubQuery.[Discounted Sales]) AS 'Discounted Sales',
                SUM(SubQuery.[Standard Sales]) AS 'Standard Sales'
                FROM (
                    SELECT CAST(
                        CASE
                        WHEN COUNT(ri.[ReceiptItemQuantity]) >= 5
                            THEN 1
                        ELSE 0
                        END AS int) AS 'Discounted Sales',
                    CAST(
                        CASE
                        WHEN COUNT(ri.[ReceiptItemQuantity]) >= 5
                            THEN 0
                        ELSE 1
                    END AS int) AS 'Standard Sales',
                    r.ReceiptId
                    FROM Receipt r
                    INNER JOIN ReceiptItem ri ON r.ReceiptId = ri.ReceiptId
                    INNER JOIN Item i ON i.ItemId = ri.ItemId
                    GROUP BY r.ReceiptId
                ) AS SubQuery
                INNER JOIN Receipt r ON SubQuery.ReceiptId = r.ReceiptId
                INNER JOIN ReceiptItem ri ON r.ReceiptId = ri.ReceiptId
                INNER JOIN Staff s ON s.StaffId = r.ReceiptStaffId
                INNER JOIN Office o  ON o.OfficeId = s.StaffOfficeId
                GROUP BY s.StaffId,s.StaffFirstName,s.StaffSurname, o.OfficeId, o.OfficeLocation
                ORDER BY [Discounted Sales]
            \end{lstlisting}

            \begin{table}[H]
                \centering
                \begin{tabular}{|l|l|l|l|}
                \hline
                StaffId & StaffFirstName & StaffSurname & OfficeId & OfficeLocation & Discounted Sales & Standard Sales \\ \hline
                S135    & Lexi           & James        & 4        & Sydney         & 312              & 98             \\ \hline
                S51     & Haley          & Taylor       & 7        & Dubbo          & 314              & 69             \\ \hline
                S17     & Daniel         & Baker        & 1        & Newcastle      & 324              & 100            \\ \hline
                S73     & John           & White        & 2        & Maitland       & 335              & 113            \\ \hline
                S161    & Jason          & Wood         & 7        & Dubbo          & 336              & 93             \\ \hline
                \end{tabular}
                \end{table}

            \subsubsection{Total Sales Value per Staff Member}

            % Consideration of the total sales per staff member was considered a highly important 
            % metric to consider also, we did consider comparing the results of this to the 
            % results of a query that did not include discount to see whom would be considered
            % the best performer if discounts were not relevant, however we also recognise this to be too
            % speclutive in nature. The required query was as follows:

            \begin{lstlisting}
                -- Sales total per staff with discounts applied ($)
                SELECT CAST(
                    CASE
                    WHEN COUNT(ri.[ReceiptItemQuantity]) >= 5
                        THEN SUM(ri.[SalePrice] * ri.[ReceiptItemQuantity]) * 0.95
                    ELSE SUM(ri.[SalePrice] * ri.[ReceiptItemQuantity])
                    END AS decimal(19,5)) AS 'Sales Totals',
                    s.StaffId,s.StaffFirstName,s.StaffSurname, o.OfficeId, o.OfficeLocation
                FROM Receipt r
                INNER JOIN ReceiptItem ri ON r.ReceiptId = ri.ReceiptId
                INNER JOIN Item i ON i.ItemId = ri.ItemId
                INNER JOIN Staff s ON s.StaffId = r.ReceiptStaffId
                INNER JOIN Customer c ON c.CustomerId = r.ReceiptCustomerId
                INNER JOIN Office o ON o.OfficeId = s.StaffOfficeId
                GROUP BY s.StaffId,s.StaffFirstName,s.StaffSurname, o.OfficeId, o.OfficeLocation
                ORDER BY 'Sales Totals' DESC;
            \end{lstlisting}

            \begin{table}[H]
                \centering
                \begin{tabular}{|l|l|l|l|}
                \hline
                Sales Total & StaffId & StaffFirstName & StaffSurname & OfficeId & OfficeLocation \\ \hline
                74137.05    & S187    & Savannah       & Jones        & 8        & Wollongong     \\ \hline
                73084.45    & S45     & Emma           & Gutierrez    & 3        & Cessnock       \\ \hline
                69981.75    & S178    & Kaitlyn        & Nguyen       & 2        & Maitland       \\ \hline
                69945.65    & S122    & Austin         & Morris       & 6        & Grafton        \\ \hline
                69875.35    & S71     & Danielle       & Myers        & 6        & Grafton        \\ \hline
                \end{tabular}
                \end{table}

            \subsubsection{Average Value Per Sale}

            % The average receipt value per staff member was another metric we considered would add
            % value to the descision to be suggested in the 
            % \hyperref[sec:Executive Summary]{\color{blue}executive summary}.
            % The required query to determine this metric was as follows:

            \begin{lstlisting}
                -- Sales average per staff with discounts applied
                SELECT (CAST(
                    CASE
                    WHEN COUNT(ri.[ReceiptItemQuantity]) >= 5
                        THEN SUM(ri.[SalePrice] * ri.[ReceiptItemQuantity]) * 0.95
                    ELSE SUM(ri.[SalePrice] * ri.[ReceiptItemQuantity])
                    END AS decimal(19,5)) / COUNT(r.ReceiptId)) AS 'Sales Average',
                    s.StaffId,s.StaffFirstName,s.StaffSurname, o.OfficeId, o.OfficeLocation
                FROM Receipt r
                INNER JOIN ReceiptItem ri ON r.ReceiptId = ri.ReceiptId
                INNER JOIN Item i ON i.ItemId = ri.ItemId
                INNER JOIN Staff s ON s.StaffId = r.ReceiptStaffId
                INNER JOIN Office o ON o.OfficeId = s.StaffOfficeId
                GROUP BY s.StaffId,s.StaffFirstName,s.StaffSurname, o.OfficeId, o.OfficeLocation
                ORDER BY 'Sales Average' DESC;
            \end{lstlisting}

            \begin{table}[H]
                    \centering
                    \begin{tabular}{|l|l|l|l|}
                    \hline
                    Sales Average & StaffId & StaffFirstName & StaffSurname & OfficeId & OfficeLocation \\ \hline
                    121.06        & S109    & Nicole         & Hernandez    & 10       & Broken Hill    \\ \hline
                    120.97        & S199    & Maria          & Smith        & 5        & Port Macquarie \\ \hline
                    120.07        & S14     & Noah           & Brooks       & 1        & Newcastle      \\ \hline
                    119.49        & S173    & Jordan         & Parker       & 5        & Port Macquarie \\ \hline
                    117.86        & S187    & Savannah       & Jones        & 8        & Wollongong     \\ \hline
                    \end{tabular}
                    \end{table}

            % \par\noindent
             
            % With results as follows:
            % We consider this to be a metric which weighs heavily in our analysis,
            % as multiple factors would impact this result, the number of items on the sale
            % (resulting in a lower total if discount was applied). Another consideration 
            % for this metric would be that it leans towards anyone who could
            % sell a larger quantity of the same item, as this lends itself towards a higher
            % receipt total. 
            
            \subsection{Best location based on Items sold and total revenue}
               
                \begin{lstlisting}
                    -- Item Count, total revenue, average revenue per item sold By Office Location
                    SELECT SUM(ri.ReceiptItemQuantity) AS ItemCount, o.OfficeId, o.OfficeLocation, SUM(ri.[SalePrice] * ri.[ReceiptItemQuantity]) * 0.95 as Revenue, Cast (SUM(ri.[SalePrice] * ri.[ReceiptItemQuantity]) * 0.95 as decimal)/SUM(ri.ReceiptItemQuantity) as AverageRevenue
                    FROM Receipt r
                    INNER JOIN ReceiptItem ri ON r.ReceiptId = ri.ReceiptId
                    INNER JOIN Staff s ON s.StaffId = r.ReceiptStaffId
                    INNER JOIN Office o on s.StaffOfficeId = o.OfficeId
                    GROUP BY o.OfficeId, o.OfficeLocation
                    ORDER BY ItemCount DESC; 
                \end{lstlisting}

                \begin{table}[H]
                    \centering
                    \begin{tabular}{|l|l|l|l|}
                    \hline
                    Item Count & OfficeId & OfficeLocation & Revenue    & AverageRevenue \\ \hline
                    96055      & 9        & Wagga Wagga    & 1744551.50 & 18.16          \\ \hline
                    65011      & 2        & Maitland       & 1200104.60 & 18.46          \\ \hline
                    64974      & 10       & Broken Hill    & 1168012.70 & 17.98          \\ \hline
                    59309      & 1        & Newcastle      & 1069951.80 & 18.04          \\ \hline
                    58754      & 4        & Sydney         & 1076412.70 & 18.32          \\ \hline
                    \end{tabular}
                    \end{table}

            \subsection{Total Number of Customers}

            \begin{lstlisting}
                
            \end{lstlisting}

            \begin{table}[H]
                \centering
                \begin{tabular}{|l|l|l|l|}
                \hline
                Sales Count & StaffId & StaffFirstName & StaffSurname \\ \hline
                \end{tabular}
                \end{table}

            \subsection{Top Team-member(s) Analysis}
                Analysis over all stores, correlation to store we want to nuke?

                \begin{lstlisting}
                   
                \end{lstlisting}

                \begin{table}[H]
                    \centering
                    \begin{tabular}{|l|l|l|l|}
                    \hline
                    Sales Count & StaffId & StaffFirstName & StaffSurname \\ \hline
                    \end{tabular}
                    \end{table}

            \subsection{Customer Analysis}
                Analysis per store - top 3?

                \begin{lstlisting}
                   
                \end{lstlisting}

                \begin{table}[H]
                    \centering
                    \begin{tabular}{|l|l|l|l|}
                    \hline
                    Sales Count & StaffId & StaffFirstName & StaffSurname \\ \hline
                    \end{tabular}
                    \end{table}

                \subsubsection{Customer Frequency}
                    Can we predict future trends in customers? 

                    \begin{lstlisting}
                        
                    \end{lstlisting}

                    \begin{table}[H]
                        \centering
                        \begin{tabular}{|l|l|l|l|}
                        \hline
                        Sales Count & StaffId & StaffFirstName & StaffSurname \\ \hline
                        \end{tabular}
                        \end{table}
                    
            \subsection{Items Per Sale}

            \begin{lstlisting}
                
            \end{lstlisting}

            \begin{table}[H]
                \centering
                \begin{tabular}{|l|l|l|l|}
                \hline
                Sales Count & StaffId & StaffFirstName & StaffSurname \\ \hline
                \end{tabular}
                \end{table}

            \subsection{Item Popularity}
                Top 3 best and worst items overall, correlation to any stores?

                \begin{lstlisting}
                   
                \end{lstlisting}

                \begin{table}[H]
                    \centering
                    \begin{tabular}{|l|l|l|l|}
                    \hline
                    Sales Count & StaffId & StaffFirstName & StaffSurname \\ \hline
                    \end{tabular}
                    \end{table}

            \subsection{Worst Performing Item}
                Correlation to store?

                \begin{lstlisting}
                    
                \end{lstlisting}

                \begin{table}[H]
                    \centering
                    \begin{tabular}{|l|l|l|l|}
                    \hline
                    Sales Count & StaffId & StaffFirstName & StaffSurname \\ \hline
                    \end{tabular}
                    \end{table}
% ------------------------------------------------------------------------------------------------ %
% CONCLUSION & RECOMMENDATIONS
% ------------------------------------------------------------------------------------------------ %
	\newpage
    \section{Conclusion and Recommendations}
           

% ------------------------------------------------------------------------------------------------ %
% REFERENCES
% ------------------------------------------------------------------------------------------------ %
    
    \newpage
    \begin{thebibliography}{9}
        \raggedright
        \bibitem{MoneyIssues}
            Reasons against TSQL Money type: Stackoverflow User; \textit{SQLMenace}
            \url{https://stackoverflow.com/questions/582797/should-you-choose-the-money-or-decimalx-y-datatypes-in-sql-server}
        \bibitem{Numeric}
            Microsoft TSQL documentation of Decimal/Numeric types
            \url{https://docs.microsoft.com/en-us/sql/t-sql/data-types/decimal-and-numeric-transact-sql?view=sql-server-2017}
        \bibitem{CTE}
        Microsoft documentation: WITH common\_table\_expression (Transact-SQL)
            \url{https://docs.microsoft.com/en-us/sql/t-sql/queries/with-common-table-expression-transact-sql?view=sql-server-2017}
        \bibitem{BusDictionairyUpselling}
        		Upselling - Business Dictionary
        		\url{http://www.businessdictionary.com/definition/upselling.html}
    \end{thebibliography}

% ------------------------------------------------------------------------------------------------ %
% APPENDIX
% ------------------------------------------------------------------------------------------------ %

    \newpage
    \section{Appendix}
    \label{sec:Appendix}
    
    \end{document}